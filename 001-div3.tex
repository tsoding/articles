\documentclass{article}
\usepackage{amsmath}
\title{Divisibility by 3}
\date{2024\\ June}
\author{Alexey Kutepov}
\begin{document}
\maketitle
\section{Intro}

In Competitive Programming there is a popular trick to figure out if a
number is divisible by 3. The number is divisible by 3 if the sum of
its digits is divisible by 3. Sounds useless, because you can just
always do \verb|x%3 == 0|, but this only works if \verb|x| is a 64
bits integer. As soon as you start working with Big Integers (integers
that don't fit into the CPU registers) you have to do something
different. And the Divisibility by 3 property is the thing you want to
do because for a sum of digits to overflow 64 bits the number should
consist of roughly over $2^{64} \div 9 = 2049638230412172401$ digits
which may require roughly 2 exabytes to store in memory (if we
allocate 1 byte per digit, you can do better but that won't help much)
so usually the limitations of the Competitive Programming problems
don't go that far (because the automatic judge systems usually
physically can't handle such large numbers too).

I always wondered why this property of integers even worked in the
first place. So I decided to prove it. And this article is my proof of
this property. There are many other proofs online, but I intentionally
didn't look into them, because I wanted to discover it myself to
expand my understand of the matter.

\section{The $\%$ Operator}

Since I'm not a mathematician I'm not gonna be using the usual mathy
Modular Arithmetic notation (primarily because after staring at the
Wikipedia article for several hours I still don't understand it). As a
Computer Programmer I'm gonna be using what is used in the majority of
Programming Languages on planet Earth: the $\%$ operator. Yes, I'm
gonna be literally using $\%$ throughout this article in the same
manner it is used in Programming Languages. If you don't like it,
that's your problem.

To effectively use the $\%$ operator we shall define it. Given Natural
($\ge 0$) integers $a$ and $c$, Positive($\ge 1$) integer $b$, where
$c < b$, $a\%b = c$ means that there is such Natural integer $k$ that

\begin{equation} \label{eq:mod-def}
  k \cdot b + c = a
\end{equation}

What it basically means is the number $a$ consists
of $k$ ``chunks'' of size $b$ plus a smaller remainder of size
$c$. The $\%$ operator enables you to extract that smaller remainder
$c$ out of $a$ given the size of the ``chunk'' $b$. If you want to
know how many chunks of size $b$ you have inside of $a$ ignoring the
remainder $c$ you can just use the integer division $\lfloor a \div b
\rfloor = k$.

%% TODO: it would be nice to have a picture of $a$ consisting of $k$
%% chunks of size $b$ plus remainder $c$

\section{Divisibility by 3 in Terms of the $\%$ Operator}

Given the definition above we can say that $a$ is divisible by $3$
when $a\%3 = 0$.

Let us have a Natural integer $n$ consisting of $m$ digits in base
10. So we can spell it like $d_md_{m-1}\dots d_{2}d_{1}$. We can
define the number $n$ in terms of its digits as the following sum:

\begin{align*}
  \sum_{i=1}^{m}d_i\cdot 10^{i-1}
\end{align*}

Which enables us to reformulate the Divisibility by 3 property like so

\begin{align} \label{eq:original-div-by-3}
  \left(\sum_{i=1}^{m}d_i\right)\%3 = 0 \implies \left(\sum_{i=1}^{m}d_i\cdot 10^{i-1}\right)\%3 = 0
\end{align}

Writting it like this gives us a pretty interesting hypothesis. What
if the actual property here is

\begin{align} \label{eq:interesting-div-by-3}
  \left(\sum_{i=1}^{m}d_i\right)\%3 = \left(\sum_{i=1}^{m}d_i\cdot 10^{i-1}\right)\%3
\end{align}

Proving (\ref{eq:interesting-div-by-3}) would imply proving (\ref{eq:original-div-by-3}) so let's just focus on exploring (\ref{eq:interesting-div-by-3}).

\section{$\%$-s can be nested in sums}

One useful property of $\%$ that will help us in our exploration can be defined as following

\begin{align} \label{eq:nest-mod}
  (a + b)\%c = (a\%c + b\%c)\%c
\end{align}

Using the definition of $\%$ operator (\ref{eq:mod-def}) we can say that

\begin{align*}
  a &= k_a\cdot c + a\%c \\
  b &= k_b\cdot c + b\%c
\end{align*}

Now let's take $(a + b)\%c$ and do a few substitutions

\begin{align*}
  (a + b)\%c
      &= (k_a\cdot c + a\%c + k_b\cdot c + b\%c)\%c \\
      &= ((k_a + k_b)\cdot c + a\%c + b\%c)\%c
\end{align*}

According to (\ref{eq:mod-def}) it doesn't really matter what $k_a +
k_b$ is equal to (as along as it's $\ge 0$) the result of $((k_a +
k_b)\cdot c + a\%c + b\%c)\%c$ will be the same even if $k_a + k_b =
0$.

\begin{align*}
  (a + b)\%c = ((k_a + k_b)\cdot c + a\%c + b\%c)\%c = (a\%c + b\%c)\%c
\end{align*}

QED.

\section{$\times$10 does not affect Divisibility by 3}

Equality (\ref{eq:interesting-div-by-3}) hints us that

\begin{align} \label{eq:remove-10}
  (a \cdot 10)\%3 = a\%3
\end{align}

You can view $a \cdot 10$ as a sum

\begin{align*}
  a \cdot 10 = \sum_{i=1}^{10}a
\end{align*}

And using (\ref{eq:nest-mod}) we can say

\begin{align*}
  (a \cdot 10)\%3 = \left(\sum_{i=1}^{10}a\right)\%3 = \left(\sum_{i=1}^{10}a\%3\right)\%3
\end{align*}

Since there are only 3 possible values of $a\%3$ ($0$, $1$ and $2$) we can just check all of them.

If $a\%3 = 0$

\begin{align*}
    (a \cdot 10)\%3 = \left(\sum_{i=1}^{10}a\%3\right)\%3 = \left(\sum_{i=1}^{10}0\right)\%3 = 0
\end{align*}

If $a\%3 = 1$

\begin{align*}
    (a \cdot 10)\%3 = \left(\sum_{i=1}^{10}a\%3\right)\%3 = \left(\sum_{i=1}^{10}1\right)\%3 = 10\%3 = 1
\end{align*}

If $a\%3 = 2$

\begin{align*}
    (a \cdot 10)\%3 = \left(\sum_{i=1}^{10}a\%3\right)\%3 = \left(\sum_{i=1}^{10}2\right)\%3 = 20\%3 = 2
\end{align*}

QED.

\section{Proof}

Let's take the definition of $n$ in terms of its digits from
(\ref{eq:interesting-div-by-3})

\begin{align*}
  \left(\sum_{i=1}^{m}d_i\cdot 10^{i-1}\right)\%3
\end{align*}

Let's apply (\ref{eq:nest-mod})

\begin{align*}
  \left(\sum_{i=1}^{m}(d_i\cdot 10^{i-1})\%3\right)\%3
\end{align*}

By continuesly applying (\ref{eq:remove-10}) we get

\begin{align*}
  \left(\sum_{i=1}^{m}d_i\%3\right)\%3
\end{align*}

By applying (\ref{eq:nest-mod}) in reverse we get

\begin{align*}
  \left(\sum_{i=1}^{m}d_i\right)\%3
\end{align*}

QED.

\section{Conclusion}

I think the main reason this property works in the first place is
because of (\ref{eq:remove-10}). You can even view
(\ref{eq:remove-10}) as sort of the essence (the core) of the
Divisibility by 3 property which I find pretty insightful.

Maybe this proof is silly and verbose compared to the other proofs out
there. I don't actually know, because I never looked into them. I'm
also not a mathematician.

\end{document}
